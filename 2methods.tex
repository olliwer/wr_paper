\chapter{Research Design}
\label{chapter:Research Design} 

This chapter elaborates the used research and analysis methods, presents interview themes and structures, and evaluates the research. 

\subsection{Interview Themes}
The interview is a semi-structured thematic interview with a standardized set of open-ended questions, which allows deep exploration of studied objects. The interviews are performed once per interviewee, with one to two interviewees per participating company. The interview themes are presented in the appendices.

\subsection{Case Study}
Case study is a way to collect data through observation and to test theories in an unmodified setting. The linkage with empirical evidence makes case study research likely to have strengths such as novelty, testability and empirical validity \cite{eisenhardt1989building}. Case studies can be divided into four different purposes: Exploratory, Descriptive, Explanatory and Improving \cite{robson2016real}. An exploratory case study seeks new insights, finds out what is happening, and generates ideas and hypotheses for new research. Descriptive case study attempts to portray a situation or a phenomenom. An explanatory case study attempts to seek an explanation to a problem or a situation, mostly in a causal relationship. Finally, an improving case study tries to improve an aspect of the studied phenomenom.

Different approaches to the order of data collection, analysis and generalization can be categorized into inductive, deductive and abductive approaches \cite{dubois2002systematic}. An inductive approach moves from data to theory, and begins by first collecting the data, and then looking for patterns and forming theories that explain those patterns. A deductive approach is in reverse order. An abductive approach starts with the consideration of facts or particular observations, which are then used to form hypotheses that relate them to a fact or a rule that accounts for them. The facts are therefore correlated into a more general description, that relates them to a wider context.

Sampling

Structure \cite{eisenhardt1989building}

Triangulation

This study is an exploratory inductive study, 

\subsection{Qualitative research}




\subsubsection{Data collection}



\subsubsection{Data analysis}

Data analysis is based on template analysis, which is a way to thematically organize and analyze qualitative data \cite{king1998template}. 

%Metodit
%	- Kvalitatiivinen case-tutkimus tuettuna kvantitatiivisella kyselyllä
%	- Kuinka kerätään dataa, kuinka analysoidaan dataa

